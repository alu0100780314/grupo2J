%%%%%%%%%%%%%%%%%%%%%%%%%%%%%%%%%%%%%%%%%%%%%%%%%%%%%%%%%%%%%%%%%%%%%%%%%%%%%%%
% Chapter 2: Fundamentos Te�ricos 
%%%%%%%%%%%%%%%%%%%%%%%%%%%%%%%%%%%%%%%%%%%%%%%%%%%%%%%%%%%%%%%%%%%%%%%%%%%%%%%

%++++++++++++++++++++++++++++++++++++++++++++++++++++++++++++++++++++++++++++++

Podemos decir que uno de los problemas m�s frecuentes en las matem�ticas es el c�lculo del �rea que obtiene a 
apartir de una funcion f(x), el eje x y los l�mites a y b.Por ejemplo, queremos calcular el �rea de la figura 1, 
recaslcando que dicha �rea est� comprendido entre los l�mites a y b, y por debajo de la funcion f(x):
 
 %aqui va el includegrafic de la fichura 1

Partiendo de que conocemos la funci�n y los valores de a y b: b lo consideramos como el l�mite superior y a como 
el l�mite inferior. Podemos obtener dos tipos de soluciones:

\begin{itemize}
\item Soluciones algebraicas: obtenemos una f�rmula exacta y precisa del �rea solicitada.
\item Soluciones num�ricas: calculamos de manera num�rica una estimaci�n o aproximacion del �rea.
\end{itemize}

Respecto a estos tipos de soluciones sabemos de antemano que las solcuiones algebraicas son mejores que las num�ricas, dado que son exactas. Pero se da el caso de que a veces, la complejidad de las funciones hace dif�cil la obtenci�n de la solcui�n algebraica, por lo que la solci�n num�rica nos permite ahorrar tiempo. Como ejemplo de esta aproximaci�n tenemos el M�todo de Simpson:

%++++++++++++++++++++++++++++++++++++++++++++++++++++++++++++++++++++++++++++++

\section{Primer apartado del segundo cap�tulo}
\label{2:sec:1}
  Primer p�rrafo de la primera secci�n.

\section{Segundo apartado del segundo cap�tulo}
\label{2:sec:2}
  Primer p�rrafo de la segunda secci�n.

