%%%%%%%%%%%%%%%%%%%%%%%%%%%%%%%%%%%%%%%%%%%%%%%%%%%%%%%%%%%%%%%%%%%%%%%%%%%%%%%
% Chapter 3: Procedimiento experimental 
%%%%%%%%%%%%%%%%%%%%%%%%%%%%%%%%%%%%%%%%%%%%%%%%%%%%%%%%%%%%%%%%%%%%%%%%%%%%%%%

  A continuaci�n se proceder� a describir detalladamente el experimento llevado a cabo.
Se hablar� de en qu� consiste exactamente, en qu� se basa y como se ha planteado su
resoluci�n. Seguidamente, se enumerar� el material necesario para realizar la prueba.
Otro apartado estar� dedicado a mencionar los resultados que se han obtenido sin realizar
ninguna objeci�n sobre la implicaci�n de los mismos. Por �ltimo, se har� uso de los
conocimientos expuestos en el cap�tulo 2 para poder analizar los datos que hemos
obtenido y, as� mismo, lo que ello supone.

%++++++++++++++++++++++++++++++++++++++++++++++++++++++++++++++++++++++++++++++
\section{Descripci�n de los experimentos}
\label{3:sec:1}

bla, bla, etc. 

%++++++++++++++++++++++++++++++++++++++++++++++++++++++++++++++++++++++++++++++
\section{Descripci�n del material}
\label{3:sec:2}
  En cuesti�n al material empleado para realizar el experimento se ha utilizado un
computador con un procesador Intel(R) Core(TM)2 Quad CPU Q6600 @ 2.40GHz 2.39 GHz y 
una memoria RAM de 3,00 GB. El sistema operativo que conten�a el computador era 
Windows 7 Home Premium de 32 bits. Para escribir los c�digos fuente de LaTeX se
usaron los programas Texmaker y Kate (versiones 3.2 y 3.8.5, respectivamente) y para 
el c�digo fuente de Python solamente se utiliz� Kate. Adem�s, para registrar los cambios
realizados en el trabajo y subirlos a GitHub se hizo uso de Git (1.7.9.5). Como 
compiladores se emple� el int�rprete de Python y el compilador de Tex.

%++++++++++++++++++++++++++++++++++++++++++++++++++++++++++++++++++++++++++++++
\section{Resultados obtenidos}
\label{3:sec:3}

bla, bla, etc. 


%------------------------------------------------------------------------------
\begin{figure}[!th]
\begin{center}
\includegraphics[width=0.75\textwidth]{mem/images/figura1.eps}
\caption{Ejemplo de figura}
\label{fig:1}
\end{center}
\end{figure}
%------------------------------------------------------------------------------


%------------------------------------------------------------------------------
%--------------------------------------------------------------------------
\begin{table}[!ht]
\begin{center}
\begin{tabular}{|c|c|c|} \hline 
\textbf{Valores de x} & \textbf{Y(funcion)} & \textbf{Y(parabola)} \\ \hline \hline
1.0000 & 0.2689 & 0.2689
\\
\hline

1.2631 & 0.2204 & 0.2338
\\
\hline

1.5263 & 0.1785 & 0.2011
\\
\hline

1.7895 & 0.1431 & 0.1707
\\
\hline

2.0526 & 0.1137 & 0.1427
\\
\hline

2.3158 & 0.0898 & 0.1169
\\
\hline

2.5789 & 0.0705 & 0.0935
\\
\hline

2.8421 & 0.0551 & 0.0725
\\
\hline

3.1052 & 0.0428 & 0.0538
\\
\hline

3.3684 & 0.0333 & 0.0374
\\
\hline

3.6316 & 0.0258 & 0.0233
\\
\hline

3.8947 & 0.0199 & 0.0116
\\
\hline

4.1579 & 0.0154 & 0.0023
\\
\hline

4.4211 & 0.0119 & -0.0048
\\
\hline

4.6842 & 0.0092 & -0.0095
\\
\hline

4.9474 & 0.0071 & -0.0119
\\
\hline

5.2105 & 0.0054 & -0.0120
\\
\hline

5.4737 & 0.0042 & -0.0096
\\
\hline

5.7368 & 0.0032 & -0.0050
\\
\hline

6.0000 & 0.0025 & 0.0025
\\
\hline
\end{tabular}
\end{center}
\caption{Valores de x escogidos para la representacion y sus imagenes correspondientes para la funcion y la parabola.}
\label{tab:1}
\end{table}


%------------------------------------------------------------------------------

%++++++++++++++++++++++++++++++++++++++++++++++++++++++++++++++++++++++++++++++
\section{An�lisis de los resultados}
\label{3:sec:4}

bla, bla, etc. 

