%%%%%%%%%%%%%%%%%%%%%%%%%%%%%%%%%%%%%%%%%%%%%%%%%%%%%%%%%%%%%%%%%%%%%%%%%%%%%
% Chapter 1: Motivaci�n y Objetivos 
%%%%%%%%%%%%%%%%%%%%%%%%%%%%%%%%%%%%%%%%%%%%%%%%%%%%%%%%%%%%%%%%%%%%%%%%%%%%%%%

	El objetivo principal de este trabajo es implementar un programa en Python que sea capaz
de resolver el problema del c�lculo del �rea bajo una funci�n conocida en un intervalo cerrado.

	Para llevar a cabo este proyecto debemos hacer uso de la integraci�n num�rica utilizando el m�todo de 
Simpson, teniendo en cuenta los siguientes aspectos:
\begin{enumerate}
  \item Debemos comprender los conceptos b�sicos de la integraci�n aproximada empleando el m�todo de Simpson.
  \item Debemos valorar el error producido entre el valor real y la aproximaci�n obtenida.
  \item Debemos analizar las representaciones gr�ficas de la funci�n estudiada y de la par�bola hallada por el m�todo de Simpson.
\end{enumerate}

%---------------------------------------------------------------------------------
\section{Objetivos espec�ficos}
\label{1:sec:1}
	Se realizar� el experimento con una funci�n conocida y definida en intervalos cerrados.
	
La funci�n dada es: 

\[f(x) = \frac{1}{1+e^{x}}, \quad x \in [1, 6]\]



