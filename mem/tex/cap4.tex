%%%%%%%%%%%%%%%%%%%%%%%%%%%%%%%%%%%%%%%%%%%%%%%%%%%%%%%%%%%%%%%%%%%%%%%%%%%%%
% Chapter 4: Conclusiones y Trabajos Futuros 
%%%%%%%%%%%%%%%%%%%%%%%%%%%%%%%%%%%%%%%%%%%%%%%%%%%%%%%%%%%%%%%%%%%%%%%%%%%%%%%

\begin{itemize}

	\item El m\'etodo de Simpson es muy \'util para funciones con una curva similar a una par\'abola en un intervalo cerrado.

	\item El m\'etodo de Simpson produce una mala aproximaci\'on si la funci\'on en el intervalo cerrado presenta una recta o m\'as de una curva.
	
	\item El m\'etodo de Simpson facilita en gran medida la integraci\'on de funciones muy dif\'iciles de resolver por otros m\'etodos.

	\item El m\'etodo de Simpson es f\'acil de aplicar ya que solo hace falta utilizar una f\'ormula y hallar las im\'agenes de la funci\'on en tres puntos diferentes.
	
	\item El m\'etodo de Simpson puede ser utilizado para comprobar el resultado de una integraci\'on y sabemos que el error es m\'inimo.
	
	\item El m\'etodo de Simpson es algo complicado de demostrar si no se tienen ciertos conocimientos matem\'aticos, por eso suele ser un m\'etodo m\'as sistem\'atico.

\end{itemize}
