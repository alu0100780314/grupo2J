%%%%%%%%%%%%%%%%%%%%%%%%%%%%%%%%%%%%%%%%%%%%%%%%%%%%%%%%%%%%%%%%%%%%%%%%%%%%%
% Chapter 4: Conclusiones y Trabajos Futuros 
%%%%%%%%%%%%%%%%%%%%%%%%%%%%%%%%%%%%%%%%%%%%%%%%%%%%%%%%%%%%%%%%%%%%%%%%%%%%%%%

\begin{itemize}

	\item El método de Simpson es muy útil para funciones con una curva similar a una parábola en un intervalo cerrado.

	\item El método de Simpson produce una mala aproximacion si la funcion en el intervalo cerrado presenta una recta o más de una curva.
	
	\item El método de Simpson facilita en gran medida la integracion de funciones muy difíciles de resolver por otros métodos.

	\item El método de Simpson es fácil de aplicar ya que solo hace falta utilizar una formula y hallar las imágenes de la funcion en tres puntos diferentes.
	
	\item El método de Simpson puede ser utilizado para comprobar el resultado de una integracion y sabemos que el error es mínimo.
	
	\item El método de Simpson es algo complicado de demostrar si no se tienen cierto conocimientos matemáticos, por eso suele ser un método más sistemático.

\end{itemize}